\documentclass{article}
\usepackage[utf8]{inputenc}

\title{Maximizing education while minimizing COVID-19 risk: priorities and pitfalls for reopening schools}
\author{Jamie Cohen, Dina Mistry, Cliff Kerr and Dan Klein }
\date{August 2020}

\begin{document}

\maketitle

\section{Introduction}

Public K-12 schools closed due to COVID-19 transmission across the country and world. A recent study has estimated that school closures were associated with a significant decline in both COVID-19 incidence and mortality. Attention is now focused on the risks and benefits of reopening schools. The educational benefits of in-person learning are enormous, as described in a recent CDC report on the importance of reopening schools. Yet much remains uncertain about the role school-age students play in COVID-19 transmission, and how effective school-based interventions may play in preventing transmission. In this analysis, we use an agent-based model of COVID-19 transmission to evaluate the health and educational outcomes associated with various school reopening strategies. 

\section{Outline of Paper}
\begin{itemize}
    \item Intro
    
    Describe school reopening globally, lessons that frame our thinking and policy options.
    
    \item Methods
    \begin{enumerate}
        \item Model overview
        \item If schools hadn't closed (maybe a theoretical exploration?)
        \item Methods for reopening schools
        \item Reacting to cases in schools
        \item Sensitivity analyses
    \end{enumerate}
    \item Results
    \begin{enumerate}
        \item 
    \end{enumerate}
    \item Discussion / Policy Implications
    \begin{enumerate}
        \item Applications to specific states (WA, OR, others?)
        \item Limitations
    \end{enumerate}
    \item Conclusions
    \item Appendix
    \begin{enumerate}
        \item Detailed model description
        \item Schools network
        \item Calibration methods/results
    \end{enumerate}
\end{itemize}

\end{document}
